\documentclass{article}
\usepackage[a4paper]{geometry}
\geometry{lmargin=2cm, rmargin=1.5cm}
\usepackage[ngerman]{babel}
\usepackage{dirtytalk}
% Formatierungen
\usepackage[a4paper]{geometry}
\usepackage{fontspec}
\setmainfont{Ubuntu}
\usepackage[font=tiny, labelfont=bf]{caption}
\usepackage{hyperref}
\hypersetup{
	colorlinks,
	citecolor=black,
	filecolor=black,
	linkcolor=black,
	urlcolor=black
}
\widowpenalty10000
\clubpenalty10000
% Header / Footer
\usepackage{fancyhdr}
\usepackage{datetime}
\usepackage{lastpage}
\pagestyle{fancy}
\fancyhf{}
\lhead{\rightmark}
\rhead{\thepage}
% Table of contents
\usepackage{tocloft}

\cftsetindents{section}{0em}{6em}
\cftsetindents{subsection}{1em}{5em}
\cftsetindents{subsubsection}{2em}{4em}

\renewcommand\cfttoctitlefont{\hfill\Large\bfseries}
\renewcommand\cftaftertoctitle{\hfill\mbox{}}

\setcounter{tocdepth}{3}

% Math
\usepackage{amsmath}
\usepackage{amssymb}
\usepackage{amsthm}
\usepackage{array}
\usepackage{todonotes}

% Unterpunkt Stufe 4
\usepackage{titlesec}
	
\setcounter{secnumdepth}{4}
	
\titleformat{\paragraph}
{\normalfont\normalsize\bfseries}{\theparagraph}{1em}{}
\titlespacing*{\paragraph}
{0pt}{3.25ex plus 1ex minus .2ex}{1.5ex plus .2ex}

% Python Code
% Default fixed font does not support bold face
\DeclareFixedFont{\ttb}{T1}{txtt}{bx}{n}{10} % for bold
\DeclareFixedFont{\ttm}{T1}{txtt}{m}{n}{10}  % for normal

% Custom colors
\usepackage{color}
\definecolor{deepblue}{rgb}{0,0,0.5}
\definecolor{deepred}{rgb}{0.6,0,0}
\definecolor{deepgreen}{rgb}{0,0.5,0}

\usepackage{listings}

% Python style for highlighting
\newcommand\pythonstyle{\lstset{
language=Python,
basicstyle=\ttm,
otherkeywords={self},             % Add keywords here
keywordstyle=\ttb\color{deepblue},
emph={MyClass,__init__},          % Custom highlighting
emphstyle=\ttb\color{deepred},    % Custom highlighting style
stringstyle=\color{deepgreen},
frame=tb,                         % Any extra options here
showstringspaces=false            % 
}}


% Python environment
\lstnewenvironment{python}[1][]
{
\pythonstyle
\lstset{#1}
}
{}

% Python for external files
\newcommand\pythonexternal[2][]{{
\pythonstyle
\lstinputlisting[#1]{#2}}}

% Python for inline
\newcommand\pythoninline[1]{{\pythonstyle\lstinline!#1!}}
\usepackage{csquotes}
\usepackage{framed}
\linespread{1.2}
\begin{document}
\begin{titlepage}
    \begin{center}
        \Huge Abiturvorbereitung Informatik
    \end{center}
\end{titlepage}
\tableofcontents
\pagebreak
\section{Sprachen \& Automaten}
\subsection{Automaten}
\subsection{Sprachen}
\subsection{Grenzen}
\section{Berechenbarkeitstheorie}
\subsection{Entscheidbarkeit}
Eine Sprache $L$ ist entscheidbar genau dann, wenn die charakteristische Funktion $\chi$ berechenbar ist.
\begin{equation}
    \chi _L(x)=
    \begin{cases}
        1, & \text{wenn } x\in L     \\
        0, & \text{wenn } x\not\in L
    \end{cases}
\end{equation}
\subsection{Semi-Entscheidbarkeit}
Eine Sprache $L$ ist entscheidbar genau dann, wenn die partielle charakteristische Funktion $\chi$ berechenbar ist.
\begin{equation}
    \chi _L(x)=
    \begin{cases}
        1,                  & \text{wenn } x\in L \\
        \text{undefiniert}, & \text{sonst}
    \end{cases}
\end{equation}
bzw.
\begin{equation}
    \chi _L(x)=
    \begin{cases}
        0,                  & \text{wenn } x\not\in L \\
        \text{undefiniert}, & \text{sonst}
    \end{cases}
\end{equation}
\subsection{Berechenbarkeit}
Ein Problem ist berechenbar, genau dann wenn ein Algorithmus zur Lösung des Problems existiert.
\subsection{Rekursive Aufzählbarkeit}
\Rightarrow NICHT Abzählbarkeit!\\\\
Menge $M$ aus $A*$ heißt rekursiv aufzählbar, genau dann wenn
\begin{equation}
    |M|=0 \vee f:\mathbb{N}\rightarrow A* \wedge M=\{f(0), f(1),...\}
\end{equation}
Dabei muss $f$ total und berechenbar sein.\\
\newtheorem*{RaSe}{Satz}
\begin{framed}
    \begin{RaSe}
        $M$ ist semi-entscheidbar genau dann, wenn $M$ rekursiv aufzählbar ist.
        \begin{proof}
            Zweiteilig:
            \begin{enumerate}
                \item $M$ ist rekursiv aufzählbar \Rightarrow $M$ ist semi-entscheidbar:\\
                      Solange $f(n) \not= w$, inkrementiere $n$
                \item $M$ ist semi-entscheidbar \Rightarrow $M$ ist rekursiv aufzählbar:\\
                      Zähle alle $w$ aus $A*$ auf und gibt diejenigen zurück, für die $\chi_M(x) = 1$ ist.
            \end{enumerate}
            \qedhere
        \end{proof}
    \end{RaSe}
\end{framed}
\subsection{Das Wortproblem}
\Rightarrow $W\in L(G)$?\\\\
\begin{tabular}{l | l}
    Zu erkennende Sprache       & Erkennender Automat               \\\hline
    endliche Sprache            & Zyklenfreier endlicher Automat    \\
    Typ-3-Sprache               & EA (Endlicher Automat)            \\
    Typ-2-Sprache               & PDA (Kellerautomat)               \\
    Typ-1-Sprache               & LBA (linear beschränkter Automat) \\
    Turingentscheidbare Sprache & TM (Turingmaschine)               \\\hline
    Typ-0-Sprache               & nicht allg. entscheidbar
\end{tabular}
\begin{framed}
\newtheorem*{Typ-1-Entscheidbarkeit}{Satz}
\begin{Typ-1-Entscheidbarkeit}
Typ-1-Sprachen sind entscheidbar.
\begin{proof}
    Ansatz: Längenmonotonie\\
    Verfolge alle Pfade der Bildungsvorschriften solange, bis der Ausdurck länger ist als das Wort. Entweder das Wort wird dabei gefunden oder es ist nicht in der Sprache enthalten. \qedhere
\end{proof}
\end{Typ-1-Entscheidbarkeit}
\end{framed}
\subsection{Probleme}
\subsubsection{Allgemein}
Eine $n$-stellige Wortfunktion:
\begin{equation}
    f: \left(A*\right)^n \rightarrow A*; \: n \in \mathbb{N}
\end{equation}
F\"ur eine $n$-stellige Eingabe gibt es eine Ausgabe.
\subsubsection{Entscheidungsproblem}
Eine $n$-stellige Wortfunktion:
\begin{equation}
    f: \left(A*\right)^n \rightarrow\{0, 1\}
\end{equation}
F\"ur eine $n$-stellige Eingabe gibt es eine Ausgabe, wahr oder falsch.
\subsubsection{Wieviele Probleme gibt es?}
\begin{framed}
    \newtheorem*{Goedel}{Gödel'scher Unvollständigkeitssatz}
    \begin{Goedel}
        In jedem widerspruchsfreien Axiomensystem gibt es Sätze, die nicht mit den Mitteln dieses Systems bewiesen werden können. (Kurt Gödel)
    \end{Goedel}
\end{framed}
\begin{framed}
    \newtheorem*{MehrProbleme}{Satz} 
    \begin{MehrProbleme}
        Es gibt mehr Probleme als Algorithmen. 
        \begin{proof}
            Was ist ein Algorithmus? Eine Turingmaschine. Über Gödelisierung wird klar, dass sich jede Turingmaschine als natürliche Zahl darstellen lässt. Somit ist die Menge der Turingmaschinene abzählbar.\\
            Was ist ein Problem? Eine $n$-stellige Wortfunktion. Jene Worte können aus einer beliebigen Menge bzw. Sprache stammen. Wir wählen die Menge der reellen Zahlen $\mathbb{R}$. Mit Hilfe der Cantor-Diagonalisierung können wir zeigen, dass $\mathbb{R}$ überabzählbar ist. Somit gibt es mehr Worte und damit auch mehr Wortfunktionen als Algorithmen.
            \qedhere
        \end{proof}
    \end{MehrProbleme}
\end{framed}
\subsection{Turingberechenbarkeit}
s. Turingmaschine\\\\
$f: A* \rightarrow A*$ heißt turingberechenbar genau dann, wenn eine Turingmaschine $TM$ existiert, sodass für alle $w_1, w_2$ aus $A*$ gilt:
\begin{equation}
    f(w_1)=w_2\Leftrightarrow \left[TM-z_0,w_1\right]\rightarrow \left[TM-z_1,w_2\right]
\end{equation}
\begin{framed}
    \newtheorem*{Church}{Churche These}  
    \begin{Church}
        Jede im intuitiven Sinn berechenbare Funktion ist auch turingberechenbar. 
    \end{Church}
\end{framed}
\section{Praktisch Unberechenbares}
\begin{quote}
    \emph{
        Berechenbarkeit ist nur praktikabel, wenn weniger Bits als Atome im Universum und weniger Rechenzeit als die Lebensdauer der Sonne erforderlich sind.
    }
\end{quote}
\Rightarrow Exponentiell w\"achst zu schnell!
\paragraph*{Rechenzeit:}
Entspricht der Anzahl der benötigten Rechenschritte (Operationen mit konstanter Dauer)
\begin{itemize}
    \item Worst-Case: Maximum aller Rechenzeiten für denkbare Eingaben
    \item Average-Case: Erwartungswert der Rechenzeit
\end{itemize}


\subsection{Die Klassen P und NP}
\paragraph*{Klasse:}
Menge mit Zugehörigkeitskriterium.
\subsubsection{[P]olynomiell Berechenbares}
Existiert ein Algorithmus \emph{zur Lösung des Problems}, dessen Worst-Case-Laufzeit durch ein Polynom über der Problemgröße abschätzbar ist?
\paragraph*{Robustheit von P:}
Unabhängig von Rechnermodell variantenstabil:\\
$\Rightarrow$ Zahlprobleme (ZP), Entscheidungsprobleme (EP), Optimierungsprobleme (OP)\\\\
\begin{tabular}{l | l}
    Umwandlung          & Rechenzeit  \\ \hline
    ZP $\rightarrow$ EP & $O(1)$      \\
    EP $\rightarrow$ ZP & $O(log(N))$ \\
    ZP $\rightarrow$ OP & $O(N)$
\end{tabular}
\paragraph*{Beispiele}
\begin{itemize}
    \item $\in P$:
          \begin{itemize}
              \item Sortierprobleme
              \item Kürzeste Wege
          \end{itemize}
    \item $\notin P$:
          \begin{itemize}
              \item Ackermann-P\'eter-Funktion
          \end{itemize}
\end{itemize}
\subsubsection{[N]ichtdeterministisch [P]olynomiell Berechenbares}
Existiert ein Algorithmus \emph{zur Überprüfung einer potentiellen Lösung}, dessen Worst-Case-Laufzeit durch ein Polynom über der Problemgröße abschätzbar ist?
\paragraph*{Beispiele}
\begin{itemize}
    \item Hamilton-Pfad
\end{itemize}
\paragraph*{MERKE:}
$P \subseteq NP \Rightarrow P = NP\not\equiv P \subset NP$
\end{document}
